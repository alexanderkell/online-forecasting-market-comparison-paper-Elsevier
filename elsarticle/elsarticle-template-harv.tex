%% 
%% Copyright 2007-2019 Elsevier Ltd
%% 
%% This file is part of the 'Elsarticle Bundle'.
%% ---------------------------------------------
%% 
%% It may be distributed under the conditions of the LaTeX Project Public
%% License, either version 1.2 of this license or (at your option) any
%% later version.  The latest version of this license is in
%%    http://www.latex-project.org/lppl.txt
%% and version 1.2 or later is part of all distributions of LaTeX
%% version 1999/12/01 or later.
%% 
%% The list of all files belonging to the 'Elsarticle Bundle' is
%% given in the file `manifest.txt'.
%% 
%% Template article for Elsevier's document class `elsarticle'
%% with harvard style bibliographic references

%\documentclass[preprint,12pt,authoryear]{elsarticle}

%% Use the option review to obtain double line spacing
%% \documentclass[authoryear,preprint,review,12pt]{elsarticle}

%% Use the options 1p,twocolumn; 3p; 3p,twocolumn; 5p; or 5p,twocolumn
%% for a journal layout:
%% \documentclass[final,1p,times,authoryear]{elsarticle}
%% \documentclass[final,1p,times,twocolumn,authoryear]{elsarticle}
%% \documentclass[final,3p,times,authoryear]{elsarticle}
\documentclass[final,3p,times,twocolumn,numbers]{elsarticle}
%% \documentclass[final,5p,times,authoryear]{elsarticle}
%% \documentclass[final,5p,times,twocolumn,authoryear]{elsarticle}

%% For including figures, graphicx.sty has been loaded in
%% elsarticle.cls. If you prefer to use the old commands
%% please give \usepackage{epsfig}

%% The amssymb package provides various useful mathematical symbols
\usepackage{amssymb}
%% The amsthm package provides extended theorem environments
\usepackage{amsthm}

\usepackage{mhchem}
\usepackage{xcolor}
\usepackage{csvsimple,booktabs}


\usepackage{algorithm}
\usepackage{algorithmicx}
\usepackage{algpseudocode}

%% The lineno packages adds line numbers. Start line numbering with
%% \begin{linenumbers}, end it with \end{linenumbers}. Or switch it on
%% for the whole article with \linenumbers.
%% \usepackage{lineno}

\journal{Sustainable Computing: Informatics and Systems}

\begin{document}

\begin{frontmatter}

%% Title, authors and addresses

%% use the tnoteref command within \title for footnotes;
%% use the tnotetext command for theassociated footnote;
%% use the fnref command within \author or \address for footnotes;
%% use the fntext command for theassociated footnote;
%% use the corref command within \author for corresponding author footnotes;
%% use the cortext command for theassociated footnote;
%% use the ead command for the email address,
%% and the form \ead[url] for the home page:
 \title{The impact of online machine-learning methods on long-term investment decisions in electricity markets}
% \tnotetext[label1]{}
 \author{Alexander J. M. Kell}
 \ead{a.kell2@newcastle.ac.uk}
% \ead[url]{home page}
% \fntext[label2]{}
% \cortext[cor1]{}
% \address{Address\fnref{label3}}
% \fntext[label3]{}

%\title{Validating a long-term electricity market model}

%% use optional labels to link authors explicitly to addresses:
%% \author[label1,label2]{}
%% \address[label1]{}
%% \address[label2]{}

\author{A. Stephen McGough, Matthew Forshaw}

\address{School of Computing, Newcastle University, Newcastle-upon-Tyne, United Kingdom}

\begin{abstract}
%% Text of abstract

\end{abstract}
%
%%%Graphical abstract
%\begin{graphicalabstract}
%\includegraphics{grabs}
%Hello test
%\end{graphicalabstract}
%
%%%Research highlights
%\begin{highlights}
%\item Validating a model
%\item Optimisation
%\item Scenario modelling
%\end{highlights}

\begin{keyword}
%% keywords here, in the form: keyword \sep keyword
Long-Term Energy Modelling \sep Online learning \sep Machine learning \sep Market investment \sep Climate Change
%% PACS codes here, in the form: \PACS code \sep code

%% MSC codes here, in the form: \MSC code \sep code
%% or \MSC[2008] code \sep code (2000 is the default)

\end{keyword}

\end{frontmatter}

%% \linenumbers

%% main text
\section{Introduction}
\label{sec:intro}


The integration of higher proportions of intermittent renewable energy sources (IRES) in the smart grid will mean that the forecasting of electricity demand will become increasingly important. This is due to the fact that supply must mean demand at all times and that IRES are less predictable than dispatchable energy sources such as coal and combined cycle gas turbines (CCGTs) \cite{Lu1993}.

Typically, peaker plants, such as reciprocal gas engines, are used to fill fluctuations in demand, that hadn't been previously planned for. These peaker plants are typically expensive to run and have higher greenhouse gas emissions than their non-peaker counterparts \cite{Mahmood2014}. 

To reduce reliance on peaker plants, it is helpful to know how much electricity demand there will be in the future, so that more efficient plants can be used to meet this demand. To aid in this, machine learning and statistical techniques have been used to accurately predict demand based on a number of different factors and data sources, such as weather, day of the week and holidays \cite{Kell2018a, Al-Musaylh2018, Vrablecova2017, Hong2014}. 

Whilst various studies have looked at how to best predict electricity demand at various time horizons, the impact that the differing methods used have on a long-term electricity market have been studied less. In this paper, we compare a number of machine learning and statistical techniques to forecast 24 hours ahead to simulate a day-ahead market. 








%- Pros and cons of ML and online ML applied to energy networks
%- Why it's important

\section{Literature Review}
\label{sec:lit-review}

- Literature review on online machine learning and different impacts on long-term investment decisions (look for things directly similar to this work)



\section{Material}
\label{sec:material}

- Introduce online learning, machine learning and ElecSim.\\
- Should I introduce theory behind machine learning techniques? If so, just the most successful?
 
\section{Methods}
\label{sec:methods}

- Use of hyperparameter tuning, talk about time taken to train/query models.\\
- Talk about ML methods used\\
- Talk about residuals\\
- Talk about sampling from residuals and placing these errors on the day-ahead market.\\


\section{Results}
\label{sec:results}

- Results of offline learning, online machine learning shown. Include residuals and MAE,MAPE,MASE etc\\

- Results of the residuals on the output of ElecSim until 2035.\\

\section{Discussion}
\label{sec:discussion}

- Discuss the impact of this on the electricity market and global economy. Make suggestions.

\section{Conclusion}
\label{sec:conclusion}

- Summary of work and future work.

\section{Funding Sources}

This work was supported by the Engineering and Physical Sciences Research Council, Centre for Doctoral Training in Cloud Computing for Big Data [grant number EP/L015358/1].





%% The Appendices part is started with the command \appendix;
%% appendix sections are then done as normal sections
%% \appendix

%% \section{}
%% \label{}

%% If you have bibdatabase file and want bibtex to generate the
%% bibitems, please use
%%
  \bibliographystyle{elsarticle-num} 
  \section*{References}
  \bibliography{library,bib_custom}

%% else use the following coding to input the bibitems directly in the
%% TeX file.

%\begin{thebibliography}{00}
%
%%% \bibitem[Author(year)]{label}
%%% Text of bibliographic item
%
%\bibitem[ ()]{}
%
%\end{thebibliography}
\end{document}

\endinput
%%
%% End of file `elsarticle-template-harv.tex'.
	